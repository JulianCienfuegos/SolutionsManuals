\newpage
\section*{\centerline{Abstract}}
\thispagestyle{plain}
\addcontentsline{toc}{chapter}{\numberline{}Abstract}
Epithelial tissue performs many important functions in animals, such as preventing contamination, transporting gases and nutrients, and fluid secretion. Macroscopically, epithelial tissue can be thought of as the layer of an animal that separates it from the exterior world. The geometrical and topological features of epithelial tissue make it amenable to computational modeling. There are several simulation codes in existence which reproduce certain aspects of epithelial tissue morphogenesis, wound healing, and equilibration, but to the best of our knowledge only one of them is freely available to the public. Unfortunately, installation and use of this software requires expertise in a unix-like operating system and advanced knowledge of several programming languages. With this in mind, I have developed \emph{Epithelium},  a lightweight epithelial tissue simulator which compiles easily on any unix-like system, and  which can also be distributed as precompiled binaries. The code has very few dependencies, and these dependencies are likely already satisfied by the default packages installed on a Linux or Mac computer. For users with access to NVIDIA GPUs, \emph{Epithelium} comes in stable and beta parallel versions. In \emph{Epithelium} simulations are fairly easy to design and run via several configuration files, the source code is highly modularized, and the algorithms used therein are extensively documented. As such, this code is useful for reproducing previous results, and for quickly designing new computational biology experiments of epithelial tissues.  

